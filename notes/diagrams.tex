% Usage: llmk diagrams.tex
% (if you are not using TeXLive < 2021, install llmk by yourself or execute, e.g.,
%   for i in diag-pa diag-pr diag-br; pdflatex -jobname=$i diagrams.tex
% )

%+++
% sequence=["WHL1", "WHL2", "BLR", "BHL", "BHR", "QED", "EW", "HVP", "HLBL"]
%
% [programs.WHL1]
% command="lualatex"
% opts=["-jobname=diagrams_WHL1"]
%
% [programs.WHL2]
% command="lualatex"
% opts=["-jobname=diagrams_WHL2"]
%
% [programs.BLR]
% command="lualatex"
% opts=["-jobname=diagrams_BLR"]
%
% [programs.BHL]
% command="lualatex"
% opts=["-jobname=diagrams_BHL"]
%
% [programs.BHR]
% command="lualatex"
% opts=["-jobname=diagrams_BHR"]
%
% [programs.QED]
% command="lualatex"
% opts=["-jobname=diagrams_QED"]
%
% [programs.EW]
% command="lualatex"
% opts=["-jobname=diagrams_EW"]
%
% [programs.HVP]
% command="lualatex"
% opts=["-jobname=diagrams_HVP"]
%
% [programs.HLBL]
% command="lualatex"
% opts=["-jobname=diagrams_HLBL"]
%+++


\documentclass[12pt,a4paper]{article}
\usepackage{amsmath,amssymb,slashed,cancel}
\usepackage{graphicx,xcolor}
\usepackage[compat=1.1.0]{tikz-feynman}
\newcommand\w[1]{_{\mathrm{#1}}}
\usetikzlibrary{patterns}
\pgfrealjobname{diagrams}
\begin{document}
%===============================================================================
% Misho's dirty hack
%===============================================================================
\makeatletter
\pgfdeclaredecoration{sines}{initial}{
  \state{initial}[
    width=+0pt,
    next state=move,
    persistent precomputation={
      \def\tikzfeynman@cs@angle@step{30}
      \def\tikzfeynman@cs@current@angle{30}
      \pgfmathsetlengthmacro{\tikzfeynman@cs@points@per@step}{
        \pgfdecoratedinputsegmentlength
        / int(\pgfdecoratedinputsegmentlength
        / \pgfdecorationsegmentlength)
        / 360
        * \tikzfeynman@cs@angle@step}}]{}
  \state{move}[
    width=+\tikzfeynman@cs@points@per@step,
    next state=draw]{\pgfpathmoveto{\pgfpointorigin}}
  \state{draw}[
    width=+\tikzfeynman@cs@points@per@step,
    persistent postcomputation={
      \pgfmathparse{mod(\tikzfeynman@cs@current@angle+\tikzfeynman@cs@angle@step, 360)}
      \let\tikzfeynman@cs@current@angle=\pgfmathresult%
    },
  ]{
    \pgfmathparse{sin(\tikzfeynman@cs@current@angle) * \pgfmetadecorationsegmentamplitude / 2}
    \tikz@decoratepathfalse
    \pgfpathlineto{\pgfqpoint{0pt}{\pgfmathresult pt}}%
  }
  \state{final}{
    \ifdim\pgfdecoratedremainingdistance>0pt\relax\pgfpathlineto{\pgfpointdecoratedpathlast}\fi
  }
}
\makeatother
\tikzfeynmanset{/tikzfeynman/every boson@@/.style={
    /tikz/draw=none,
    /tikz/decoration={name=none},
    /tikz/postaction={
      /tikz/draw,
      /tikz/decoration={
        sines,
        meta-amplitude=1mm,
        segment length=8pt,
      },
      /tikz/decorate=true,
    }
}}

%===============================================================================
% Diagrams
%===============================================================================
% textwidth = 16 cm; 4cm x 2.46cm + 0.3cm margin
% dimension: 6x4cm
\def\tikzframe{
  \node (frame1) at (+2.0, -1.23) {};
  \node (frame2) at (+2.0, +1.23) {};
  \node (frame3) at (-2.0, -1.23) {};
  \node (frame4) at (-2.0, +1.23) {};
  \node (margin0) at (+2.3, -1.53) {};
  \node (margin2) at (+2.3, +1.53) {};
  \node (margin3) at (-2.3, -1.53) {};
  \node (margin4) at (-2.3, +1.53) {};
}
\def\gmtwodiagram#1#2#3#4#5{ %[name, upper, lower, feynman-command, other-commmand]
\beginpgfgraphicnamed{diagrams_#1}
\begin{tikzpicture}[thick]
  \tikzframe
  \coordinate (mu1) at (-2.0, -0.5);
  \coordinate (mu2) at (+2.0, -0.5);
  \coordinate (c1)  at (-1.2, -0.5);
  \coordinate (c1p) at (-0.7, -0.5);
  \coordinate (c3)  at (0,    -0.5);
  \coordinate (c2)  at (+1.2, -0.5);
  \coordinate (c2p) at (+0.7, -0.5);
  \coordinate (c0)  at (0,    +0.7);
  \coordinate (cx)     at (+0.8485, +0.3485);
  \coordinate (cxaway) at (+1.0864, +0.5364);
  \coordinate (xa1) at (-0.1, +0.6);
  \coordinate (xa2) at (+0.1, +0.8);
  \coordinate (xb1) at (+0.1, +0.6);
  \coordinate (xb2) at (-0.1, +0.8);
  \coordinate (gam) at (+1.6, +1.0);
  \node [anchor=south] at (-1.8, -0.52) {$\mu\w L$};
  \node [anchor=south] at (+1.8, -0.52) {$\mu\w R$};
  \node [anchor=south] at (+1.7, +0.89) {$\gamma$};
  \node [anchor=south] at (   0, +0.71) {#2};
  %\node [anchor=south] at (   0, -0.48) {#3}; % upper than the line
  \node [anchor=south] at (   0, -1.18) {#3};  % lower than the line
  \begin{feynman}\diagram*{#4};\end{feynman}%
  #5;%
\end{tikzpicture}
\endpgfgraphicnamed
}


\gmtwodiagram{WHL1}{$\widetilde W^\pm$--$\widetilde H^\pm$}{$\widetilde\nu_\mu$}{
  (mu1) -- [fermion] (c1),
  (c2) -- [fermion] (mu2),
  (cxaway) -- [boson] (gam); % photon
  (c1) -- [quarter left,plain,boson] (c0) -- [quarter left,plain,boson] (c2), % upper
  (xa1) -- (xa2),   (xb1) -- (xb2), % upper cross
  (c1) -- [scalar] (c2), %lower
}{}

\gmtwodiagram{WHL2}{$\widetilde\mu\w L$}{$\widetilde W^0$--$\widetilde H^0$}{
  (mu1) -- [fermion] (c1),
  (c2) -- [fermion] (mu2),
  (cx) -- [boson] (gam); % photon
  (c1) -- [quarter left,scalar] (c0) -- [quarter left,scalar] (c2), % upper
  (c1) -- [plain,boson,insertion={[size=1mm]0.5}] (c2), % lower
}{\relax}

\gmtwodiagram{BLR}{$\widetilde\mu\w L$--$\widetilde\mu\w R$}{$\widetilde B$}{
  (mu1) -- [fermion] (c1),
  (c2) -- [fermion] (mu2),
  (cxaway) -- [boson] (gam); % photon
  (c1) -- [quarter left,scalar] (c0) -- [quarter left,scalar] (c2), % upper
  (xa1) -- (xa2),   (xb1) -- (xb2), % upper cross
  (c1) -- [plain,boson] (c2), %lower
}{\relax}

\gmtwodiagram{BHL}{$\widetilde\mu\w L$}{\phantom{x}$\widetilde B$--$\widetilde H^0$}{
  (mu1) -- [fermion] (c1),
  (c2) -- [fermion] (mu2),
  (cx) -- [boson] (gam); % photon
  (c1) -- [quarter left,scalar] (c0) -- [quarter left,scalar] (c2), % upper
  (c1) -- [plain,boson,insertion={[size=1mm]0.5}] (c2), % lower
}{\relax}

\gmtwodiagram{BHR}{$\widetilde\mu\w R$}{$\widetilde H^0$--$\widetilde B$\phantom{x}}{
  (mu1) -- [fermion] (c1),
  (c2) -- [fermion] (mu2),
  (cx) -- [boson] (gam); % photon
  (c1) -- [quarter left,scalar] (c0) -- [quarter left,scalar] (c2), % upper
  (c1) -- [plain,boson,insertion={[size=1mm]0.5}] (c2), % lower
}{\relax}

\gmtwodiagram{QED}{}{}{
    (mu1) -- [fermion] (c1p);
    (c2p) -- [fermion] (mu2);
    (cx) -- [boson] (gam);
}{%
  \filldraw[fill=white](0,0) circle (0.9);
  \node [anchor=center] at (   0, 0) {QED};
}

\gmtwodiagram{EW}{}{}{
    (mu1) -- [fermion] (c1p);
    (c2p) -- [fermion] (mu2);
    (cx) -- [boson] (gam);
}{%
  \filldraw[fill=white](0,0) circle (0.9);
  \node [anchor=center] at (   0, 0) {EW};
 }

\gmtwodiagram{HVP}{}{}{
    (mu1) -- [fermion] (c1) -- [quarter left] (c0) -- [quarter left] (c2) -- [fermion] (mu2);
    (cx) -- [boson] (gam);
    (c1) -- [boson] (c2);
}{%
  \filldraw[fill=white](0,-0.5) circle (0.5);
  \filldraw[pattern=north east lines](0,-0.5) circle (0.5);
}

\gmtwodiagram{HLBL}{}{}{
    (mu1) -- [fermion] (c1) -- (c2) -- [fermion] (mu2);
    (c1) -- [boson] (c0) -- [boson] {(c2), (c3), (gam)};
}{%
  \filldraw[fill=white](0,0.5) circle (0.5);
  \filldraw[pattern=north east lines](0,0.5) circle (0.5);
}


\end{document}
